% Options for packages loaded elsewhere
\PassOptionsToPackage{unicode}{hyperref}
\PassOptionsToPackage{hyphens}{url}
%
\documentclass[
]{article}
\usepackage{lmodern}
\usepackage{amssymb,amsmath}
\usepackage{ifxetex,ifluatex}
\ifnum 0\ifxetex 1\fi\ifluatex 1\fi=0 % if pdftex
  \usepackage[T1]{fontenc}
  \usepackage[utf8]{inputenc}
  \usepackage{textcomp} % provide euro and other symbols
\else % if luatex or xetex
  \usepackage{unicode-math}
  \defaultfontfeatures{Scale=MatchLowercase}
  \defaultfontfeatures[\rmfamily]{Ligatures=TeX,Scale=1}
\fi
% Use upquote if available, for straight quotes in verbatim environments
\IfFileExists{upquote.sty}{\usepackage{upquote}}{}
\IfFileExists{microtype.sty}{% use microtype if available
  \usepackage[]{microtype}
  \UseMicrotypeSet[protrusion]{basicmath} % disable protrusion for tt fonts
}{}
\makeatletter
\@ifundefined{KOMAClassName}{% if non-KOMA class
  \IfFileExists{parskip.sty}{%
    \usepackage{parskip}
  }{% else
    \setlength{\parindent}{0pt}
    \setlength{\parskip}{6pt plus 2pt minus 1pt}}
}{% if KOMA class
  \KOMAoptions{parskip=half}}
\makeatother
\usepackage{xcolor}
\IfFileExists{xurl.sty}{\usepackage{xurl}}{} % add URL line breaks if available
\IfFileExists{bookmark.sty}{\usepackage{bookmark}}{\usepackage{hyperref}}
\hypersetup{
  hidelinks,
  pdfcreator={LaTeX via pandoc}}
\urlstyle{same} % disable monospaced font for URLs
\usepackage{longtable,booktabs}
% Correct order of tables after \paragraph or \subparagraph
\usepackage{etoolbox}
\makeatletter
\patchcmd\longtable{\par}{\if@noskipsec\mbox{}\fi\par}{}{}
\makeatother
% Allow footnotes in longtable head/foot
\IfFileExists{footnotehyper.sty}{\usepackage{footnotehyper}}{\usepackage{footnote}}
\makesavenoteenv{longtable}
\setlength{\emergencystretch}{3em} % prevent overfull lines
\providecommand{\tightlist}{%
  \setlength{\itemsep}{0pt}\setlength{\parskip}{0pt}}
\setcounter{secnumdepth}{-\maxdimen} % remove section numbering

\author{}
\date{}

\begin{document}

2.

Cross-validation aims to split the data into 5 groups: 4 training and 1
testing sets. We will iterate through the dataset and use each index as
the testing set while training the rest of the index values. The
training will produce a more accurate training and testing error
parameter. Cross-validation allows us to use multiple predictions for a
more accurate data prediction compared to traditional prediction methods
that uses singular training values. Although cross-validation may result
in extra computational steps than the traditional train-test split.

\begin{longtable}[]{@{}ll@{}}
\toprule
Degree & Test Error\tabularnewline
\midrule
\endhead
1 & 3\tabularnewline
2 & 2\tabularnewline
3 & 3\tabularnewline
4 & 3\tabularnewline
5 & 4\tabularnewline
6 & 4\tabularnewline
7 & 5\tabularnewline
8 & 3\tabularnewline
9 & 4\tabularnewline
10 & 4\tabularnewline
11 & 3\tabularnewline
12 & 4\tabularnewline
13 & 3\tabularnewline
14 & 4\tabularnewline
15 & 4\tabularnewline
16 & 4\tabularnewline
17 & 3\tabularnewline
18 & 4\tabularnewline
19 & 4\tabularnewline
20 & 5\tabularnewline
\bottomrule
\end{longtable}

Runs, Degree

0,2.0

1,3.0

2,4.0

3,3.0

4,3.0

5,3.0

6,4.0

7,4.0

8,2.0

9,4.0

10,5.0

11,3.0

12,5.0

13,3.0

14,3.0

15,4.0

16,4.0

17,3.0

18,3.0

19,4.0

\begin{longtable}[]{@{}lll@{}}
\toprule
& Test Errors & Best Degrees\tabularnewline
\midrule
\endhead
Mean & 0.102 & 3.75\tabularnewline
Standard deviation & 0.0207 & 1.34\tabularnewline
\bottomrule
\end{longtable}

\end{document}
